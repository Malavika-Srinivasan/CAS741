\documentclass[12pt, titlepage]{article}

\usepackage{amsmath, mathtools}

\usepackage[round]{natbib}
\usepackage{amsfonts}
\usepackage{amssymb}
\usepackage{graphicx}
\usepackage{colortbl}
\usepackage{xr}
\usepackage{hyperref}
\usepackage{longtable}
\usepackage{xfrac}
\usepackage{tabularx}
\usepackage{float}
\usepackage{siunitx}
\usepackage{booktabs}
\usepackage{multirow}
\usepackage[section]{placeins}
\usepackage{caption}
\usepackage{fullpage}
\usepackage{comment}


\hypersetup{
bookmarks=true,     % show bookmarks bar?
colorlinks=true,       % false: boxed links; true: colored links
linkcolor=red,          % color of internal links (change box color with linkbordercolor)
citecolor=blue,      % color of links to bibliography
filecolor=magenta,  % color of file links
urlcolor=cyan          % color of external links
}

\usepackage{array}

\externaldocument{../../SRS/CA}

%% Comments

\usepackage{color}

\newif\ifcomments\commentstrue

\ifcomments
\newcommand{\authornote}[3]{\textcolor{#1}{[#3 ---#2]}}
\newcommand{\todo}[1]{\textcolor{red}{[TODO: #1]}}
\else
\newcommand{\authornote}[3]{}
\newcommand{\todo}[1]{}
\fi

\newcommand{\wss}[1]{\authornote{blue}{SS}{#1}}
\newcommand{\an}[1]{\authornote{magenta}{Malavika}{#1}}


\newcommand{\progname}{CFS}

\begin{document}

\title{Module Interface Specification for \progname}

\author{Malavika Srinivasan}

\date{\today}

\maketitle

\pagenumbering{roman}

\section{Revision History}

\begin{tabularx}{\textwidth}{p{3cm}p{2cm}X}
\toprule {\bf Date} & {\bf Version} & {\bf Notes}\\
\midrule
Nov 10, 2018 & 1.0 & First draft by Malavika\\
Nov 21, 2018 & 1.1 & Corrections based on presentation by Malavika\\
Nov 24, 2018 & 1.2 & MIS submission draft by Malavika\\
\bottomrule
\end{tabularx}

~\newpage

\section{Symbols, Abbreviations and Acronyms}

See CA Documentation at 
\url{https://github.com/Malavika-Srinivasan/CAS741/blob/master/docs/SRS/CA.pdf}


\newpage

\tableofcontents

\newpage

\pagenumbering{arabic}

\section{Introduction}

The following document details the Module Interface Specifications for 
\progname{} (Curve Fitting Software). The goal of \progname{} is to find the 
parameters of the curve which is the best possible fit through the given set of 
`$n$' data points, where $n >= 2$.

Complementary documents include the Commonality Analysis and Module Guide. The 
full documentation and implementation can be found at the repository 
\url{https://github.com/Malavika-Srinivasan/CAS741}.

\section{Notation}

The structure of the MIS for modules comes from \citet{HoffmanAndStrooper1995},
with the addition that template modules have been adapted from
\cite{GhezziEtAl2003}. The mathematical notation comes from Chapter 3 of
\citet{HoffmanAndStrooper1995}.  For instance, the symbol := is used for a
multiple assignment statement and conditional rules follow the form $(c_1
\Rightarrow r_1 | c_2 \Rightarrow r_2 | ... | c_n \Rightarrow r_n )$.



The following table summarizes the primitive data types used by \progname. 

\begin{center}
\renewcommand{\arraystretch}{1.2}
\noindent 
\begin{tabular}{l l p{7.5cm}} 
\toprule 
\textbf{Data Type} & \textbf{Notation} & \textbf{Description}\\ 
\midrule
character & char & a single symbol or digit\\
integer & $\mathbb{Z}$ & a number without a fractional component in (-$\infty$, $\infty$) \\
natural number & $\mathbb{N}$ & a number without a fractional component in [1, $\infty$) \\
real & $\mathbb{R}$ & any number in (-$\infty$, $\infty$)\\

boolean & $\mathbb{B}$ & a value from the set\ \{$True$,$False$\}\\
\bottomrule
\\
\end{tabular} 
\end{center}

\noindent
The specification of \progname \ uses some derived data types: sequences, strings, and
tuples. Sequences are lists filled with elements of the same data type. Strings
are sequences of characters. Tuples contain a list of values, potentially of
different types. In addition, \progname \ uses functions, which
are defined by the data types of their inputs and outputs. Local functions are
described by giving their type signature followed by their specification.

\section{Module Decomposition}
\ms{Needs to be changed in MG based on Dr.\ Smith's suggestion}
The following table is taken directly from the Module Guide document for this project.

\begin{table}[h!]
	\centering
	\begin{tabular}{p{0.3\textwidth} p{0.6\textwidth}}
		\toprule
		\textbf{Level 1} & \textbf{Level 2}\\
		\midrule
		
		{Hardware-Hiding Module} & ~ \\
		\midrule
		
		\multirow{3}{0.3\textwidth}{Behaviour-Hiding Module}& Input\\
		& Output\\
		
		\midrule
		
		\multirow{3}{0.3\textwidth}{Software Decision Module} & Sequence 
		Services\\
		& Interpolation\\
		& Regression\\
		& Plot\\
		\bottomrule
		
	\end{tabular}
	\caption{Module Hierarchy}
	\label{TblMH}
\end{table}

\newpage
~\newpage
\section{MIS of Input module} \label{mInput}

This module corresponds to R\ref{R_IInputs}, R\ref{R_IverifyIPType} and 
R\ref{R_IverifyIPData} in the CA document. The secrets of 
this module are how the data points are input and how they are verified.  The 
load and verify secrets are isolated to their own access programs.

\subsection{Module}

input

\subsection{Uses} 

Sequence services(\ref{seqSer})

\subsection{Syntax}

\begin{tabular}{p{3cm} p{3cm} p{4cm} >{\raggedright\arraybackslash}p{5cm}}
	\toprule
	\textbf{Name} & \textbf{In} & \textbf{Out} & \textbf{Exceptions} \\
	\midrule
	loadInput & inFile : string & - &  FileError \\
	verifyInput & - & - & ValueError, TypeError  \\
	verifyDegree & degree:$\mathbb{N}$ & $degVerify$:$\mathbb{B}$ & ValueError\\
	\bottomrule
\end{tabular}

\subsubsection{Exported Constants}

NA

\subsubsection{Exported Access Programs}

\begin{center}
	\begin{tabular}{p{2cm} p{4cm} p{4cm} p{2cm}}
		\hline
		\textbf{Name} & \textbf{In} & \textbf{Out} & \textbf{Exceptions} \\
		\hline
		verifyDegree & degree:$\mathbb{N}$ &  $degVerify$:$\mathbb{B}$  & 
		ValueError \\
		\hline
	\end{tabular}
\end{center}

\ms{I need the verifyDegree as an exported access program because I will be 
verifying the degree input from regression module if necessary.}

\wss{You don't have a Syntax section and an Exported Access Programs.  The
  syntax section is already listing all access programs that are exported.}

\subsection{Semantics}

\subsubsection{State Variables}

\renewcommand{\arraystretch}{1.2}
\begin{longtable*}[l]{l} 
	\# From R\ref{R_IInputs} in CA\\
	$t$: $\mathbb{R}^n$ \\
	$y$: $\mathbb{R}^n$ \\	
	
\end{longtable*}

\subsubsection{Environment Variables}


inFile: sequence of string \ms{I am leaving it as sequence of strings because 
sometimes, you may want to append a file name to a path name.}
\ms{Also, I am thinking if I really need a file input. I think this should be 
the user program's responsibility. My access programs do not have an interface 
to accept filename. They will accept only x and y arrays. This needs to be 
discussed. } 

\wss{The environment variable for a file is not the name of the file, but the
  contents of the file.  You need to decide what your design is.  I see the
  input module as providing services that you can use.  Since your module has
  state information, you need a way to get this state information into the
  module.  A file is a logical choice.}
\wss{You cannot use the same name for your environment variable as you use for
  the first field of your loadInput method.  It is confusing.  I think this
  might be part of why you are confusing the file name and the file contents.}

\subsubsection{Assumptions}

\begin{itemize}
	
	\item The loadInput will be called before calling the verifyInput 
	method.
	\item Regression module will be called before calling verifyDegree method.
	\item The loadInput will be called before the values of any state variables 
	will be accessed.
	
\end{itemize}
\subsubsection{Access Routine Semantics}


\noindent \textbf{input.$t$:} \wss{You don't actually list an access program
  with this name?}

\begin{itemize}
	\item transition: NA 	
	\item output: \textit{out} := $t$
	\item exception: NA
\end{itemize}

\noindent \textbf{input.$y$:} \wss{You don't actually list an access program
  with this name?}
\begin{itemize}
	\item transition: NA 	
	\item output: \textit{out} := $y$
	\item exception: NA
\end{itemize}

\noindent \textbf{loadInput($s$):}
\begin{itemize}
	\item transition: {inFile} is used to modify the state variables using 
	the following procedural specification:
	
	\begin{enumerate}
		
		\item Read data sequentially from inputFile to populate the state 
		variables $t$ and $y$.
		
		\item verifyInput()
		
	\end{enumerate}
	\item output: NA
	\item exception: $exc$ := a file name $s$ cannot be found $\lor$ the format 
	of inFile is incorrect $\Rightarrow$ FileError.\ms{I am yet to decide if 
	the 
	library will take a .txt and .csv or both. This specification will become 
	formal once we decide the type of the input file.}
	
\end{itemize}

\noindent \textbf{verifyInput():}
\begin{itemize}
\item transition:NA
\item output: NA
\item exception: $exc$ := 
\end{itemize}

\begin{tabular}{p{12cm} p{4.75cm}}
	
	\toprule
	\textbf{Name}&\textbf{Exception}\\
	\midrule


$ (|t| \leq 1 \lor |y| \leq 1)$ & $\Rightarrow$ LengthError\\
$ (|t| \neq |y|)$ & $\Rightarrow$ LengthMismatchError\\
$ (\forall x \in t \land  x \notin \mathbb{R} )$ & $\Rightarrow$ TypeError\\
$ (\forall x \in y \land  x \notin \mathbb{R} )$ & $\Rightarrow$ TypeError\\


\bottomrule
\end{tabular}\\

~\newline
\noindent \textbf{verifyDegree():}
\begin{itemize}
	\item transition: 
	\begin{itemize}
		
		\item $(deg \in \mathbb{N})\Longrightarrow $degVerify$ = 
		True$
	\end{itemize}
	\item output: 
	\item exception: $exc$ := 
\end{itemize}

\begin{tabular}{p{12cm} p{4.75cm}}
	
	\toprule
	\textbf{Name}&\textbf{Exception}\\
	\midrule
	
	
	$ (deg \notin \mathbb{N})$ & $\Rightarrow$ ValueError\\

	
	\bottomrule
\end{tabular}


\subsubsection{Local Functions}

NA


%###########################################################################


~\newpage

\section{MIS of Interpolation module} \label{mInterp}

This module corresponds to R\ref{R_Int_or_Reg}, R\ref{R_Imethod}, 
R\ref{R_Polynomialmethod}, R\ref{R_Piecewisemethod} and R\ref{R_ICalculate}
in section \ref{FReq} of CA document.


\subsection{Module}

interp

\subsection{Uses}

Input(\ref{mInput}), Output(\ref{mOutput}), Sequence services(\ref{seqSer})


\subsection{Syntax}

\subsubsection{Exported Constants}


NA

\subsubsection{Exported Access Programs}

\begin{center}
\begin{tabular}{p{4cm} p{4cm} p{4cm} p{3cm}}
\hline
\textbf{Name} & \textbf{In} & \textbf{Out} & \textbf{Exceptions} \\
\hline
interpMonomial & t:$\mathbb{R}^n$,y:$\mathbb{R}^n$  & - & - \\
interpLagrange & t:$\mathbb{R}^n$,y:$\mathbb{R}^n$ & - & - \\
interpNewton & t:$\mathbb{R}^n$,y:$\mathbb{R}^n$ & - & - \\
interpHermiteCubic & t:$\mathbb{R}^n$,y:$\mathbb{R}^n$ & - & 
- \\
interpBSpline & t:$\mathbb{R}^n$,y:$\mathbb{R}^n$ & - & - \\

evalMonomial & t:$\mathbb{R}^n$,y:$\mathbb{R}^n$ & yNew:$\mathbb{R}^n$  & - \\
evalLagrange & t:$\mathbb{R}^n$,y:$\mathbb{R}^n$ & yNew:$\mathbb{R}^n$ & - \\
evalNewton & t:$\mathbb{R}^n$,y:$\mathbb{R}^n$ & yNew:$\mathbb{R}^n$ & - \\
evalHermiteCubic & t:$\mathbb{R}^n$,y:$\mathbb{R}^n$ & yNew:$\mathbb{R}^n$ & - 
\\
evalBSpline & t:$\mathbb{R}^n$,y:$\mathbb{R}^n$ & yNew:$\mathbb{R}^n$ & - \\
\hline
\end{tabular}
\end{center}

\wss{Your input module has state information, yet you are explicitly providing
  inputs of this same information.  You need to decide what your design is and
  stick to it.}

\subsection{Semantics}

\subsubsection{State Variables}
\begin{itemize}
	\item x:$\mathbb{R}^{m X n}$, \text{where $n$ is the number of data points 
	and $m$ is the number of sections.}\\
	
	\item interval:$\mathbb{R}^m$, \text{$m$ is the number of sections} \\
	
\end{itemize}

\wss{Why do you have a state variable?  The connection between your inputs and
  your outputs is unclear.  Is the state variable keeping the information that
  you need for evaluation?  That would make sense.  I think your design would
  make more sense if you followed the design we used for SFS.  You can have one
  interpolating polynomial object and the specific method it uses to build its
  state variables comes as an input parameter.  This would be similar to the
  2AA4/2ME3 assignment I showed in class, where the order of interpolation was a
  parameter.  This would mean that you only needed one eval.  The object itself
  would know how to interpret eval}

\subsubsection{Environment Variables}

NA

\subsubsection{Assumptions}

\begin{itemize}
	\item evalMonomial() will be called after interpMonomial().
	\item evalLagrange() will be called after interpLagrange().
	\item evalNewton() will be called after interpNewton().
	\item evalHermiteCubic() will be called after interpHermiteCubic.
	\item evalBSpline() will be called after interpBSpline().
\end{itemize}

\subsubsection{Access Routine Semantics}


\noindent \textbf{interp.$x$:}
\begin{itemize}
	\item transition: NA	
	\item output: \textit{out} := $x$
	\item exception: NA
\end{itemize}
%%%%%%%%%%%%%%%%%%%%%%%%%%%%%%%%%%%%%%%%%%%%%%%%%%%%%%%%%%%%%%%%%%%%%%%%%
~\newline
~\newline
\noindent \textbf{interpMonomial($t$,$y$):}
\begin{itemize}
\item transition: 
\begin{itemize}
	\item 
	\begin{enumerate}
	\item x: $\mathbb{R}^{m X n} = 0.0 $, this is initialization. This is 
	necessary as all the methods will be using the same state variable.
	
	\item x: $\mathbb{R}^{m X n}$ obtained by solving for $x$ using the 
	equation below.
	
	\begin{equation*}
	\begin{bmatrix}
	1 & t_{1} & t_{1} ^2 & \dots & t_{1} ^{n-1} \\
	1 & t_{2} & t_{2} ^2 & \dots & t_{2} ^{n-1} \\
	\vdots & \vdots & \vdots & \vdots \\
	1 & t_{n} & t_{n} ^2 & \dots & t_{n} ^{n-1} \\
	\end{bmatrix}
	\begin{bmatrix}
	x_1  \\
	x_2 \\
	\vdots \\
	x_n \\
	\end{bmatrix} = 
	\begin{bmatrix}
	y_1  \\
	y_2 \\
	\vdots \\
	y_n \\
	\end{bmatrix}
	\end{equation*}\\
	\end{enumerate}
	\item $interval[0] \coloneqq t[0] $ , this is initialized to 
	starting point for non piecewise functions.
\end{itemize}
  
\item output: NA
 
\item exception: NA
\end{itemize}

\noindent \textbf{evalMonomial($t$,$y$):}
\begin{itemize}
	\item transition: NA
	
	\item output:  \textit{out} := $yNew$
	
	$\forall t_i \in t$, yNew.append(y1) where y1 is obtained as shown below.
	\begin{equation*}
	y1 = x_1 + x_2 t_i + x_3 t_i ^2 + x_4 t_i ^3 + ... x_n t_i ^{n-1}  
	\end{equation*}
	
	\item exception: NA
\end{itemize}

%%%%%%%%%%%%%%%%%%%%%%%%%%%%%%%%%%%%%%%%%%%%%%%%%%%%%%%%%%%%%%%%%%%%%%%%%

\noindent \textbf{interpLagrange($t$,$y$):}
\begin{itemize}
	\item transition:
	\begin{itemize}
		\item 
		\begin{enumerate}
			\item x: $\mathbb{R}^{m X n} = 0.0 $, this is initialization. 
			This is necessary as all the methods will be using the same state 
			variable.
			
			\item x: $\mathbb{R}^{m X n}$ obtained by solving for $x$ using 
			the equation below.
			
			\begin{equation*}
			\begin{bmatrix}
			1 & 0 & 0 & \dots & 0 \\
			0 & 1 & 0 & \dots & 0 \\
			\vdots & \vdots & \vdots & \vdots \\
			0 & 0 & 0 & \dots & 1 \\
			\end{bmatrix}
			\begin{bmatrix}
			x_1  \\
			x_2 \\
			\vdots \\
			x_n \\
			\end{bmatrix} = 
			\begin{bmatrix}
			y_1  \\
			y_2 \\
			\vdots \\
			y_n \\
			\end{bmatrix}
			\end{equation*}\\ \\ 
		\end{enumerate}
		\item $interval[0] \coloneqq t[0] $ , this is initialized to 
		starting point for non piecewise functions.
	\end{itemize}	
	\item output: NA 
	
	\item exception: NA
\end{itemize}

\noindent \textbf{evalLagrange($t$,$y$):}
\begin{itemize}
	\item transition: NA
	
	\item output:  \textit{out} := $yNew$
	
	$\forall t_i \in t$, yNew.append($y_0$) where $y_0$ is obtained as shown 
	below.
	
	\begin{equation*}
	y_0 = y_1 l_1 (t) + y_2 l_2(t) + \dots y_n l_n (t).
	\end{equation*}
	where $l_j (t)$ is given by,
	
	
	\begin{equation*}
	l_j (t) =  \frac{ \Pi _{k=1, k\neq j} ^n (t - t_k)} {\Pi _{k=1,k \neq j} ^n 
		(t_j - t_k)}
	\end{equation*}
	\item exception: NA
\end{itemize}


%%%%%%%%%%%%%%%%%%%%%%%%%%%%%%%%%%%%%%%%%%%%%%%%%%%%%%%%%%%%%%%%%%%%%%%%%
~\newline
~\newline

\noindent \textbf{interpNewton($t$,$y$):}
\begin{itemize}
	\item transition: 
	\begin{itemize}
		\item 
		\begin{enumerate}
			\item x: $\mathbb{R}^{m X n} = 0.0 $, this is initialization. 
			This is necessary as all the methods will be using the same state 
			variable.
			
			\item x: $\mathbb{R}^{m X n}$ obtained by solving for $x$ using 
			the equation below.
			\begin{equation*}
			\begin{bmatrix}
			\pi_0 (t_0) & 0           & 0            & \dots         & 0 \\
			\pi_0 (t_1) & \pi_1 (t_1) & 0            & \dots         & 0 \\
			\vdots      & \vdots      & \vdots       &\ddots         & 0 \\
			\pi_0 (t_n) & \pi_1 (t_n) & \pi_2 (t_n)  & \dots         & \pi_n 
			(t_n) \\
			\end{bmatrix}
			\begin{bmatrix}
			x_1  \\
			x_2 \\
			\vdots \\
			x_n \\
			\end{bmatrix} = 
			\begin{bmatrix}
			y_1  \\
			y_2 \\
			\vdots \\
			y_n \\
			\end{bmatrix} 
			\end{equation*}\\ 
			
			where,
			
			$i < j$ $\implies$ $\pi_j (t) = 0$, 
			$i \geq j$  $\implies$  $\pi(t) = \Pi_{k=1}^{j-1} (t - t_k)$  
			
			%When the limits make it vacuous, the product is taken to be 1.\\
			%\ms{work on vacuous}
		\end{enumerate}
		\item $interval[0] \coloneqq t[0] $ , this is initialized to 
		starting point for non piecewise functions.
	\end{itemize}	
	\item output: NA	
\item exception: NA
\end{itemize}




\noindent \textbf{evalNewton($t$,$y$):}
\begin{itemize}
	\item transition: NA
	
	\item output:  \textit{out} := $yNew$
	
	$\forall t_i \in t$, yNew.append($y_0$) where $y_0$ is obtained as shown 
	below.
	
	\begin{equation*}
	y_0 = x_1 + x_2(t-t_1) + x_3(t-t_1)(t-t_2) + ... 
	x_n(t-t_1)(t-t_2)...(t-t_{n-1})
	\end{equation*}

	\item exception: NA
\end{itemize}



%%%%%%%%%%%%%%%%%%%%%%%%%%%%%%%%%%%%%%%%%%%%%%%%%%%%%%%%%%%%%%%%%%%%%%%%%
\noindent \textbf{interpHermiteCubic($t$,$y$):}
\begin{itemize}
	\item transition: 
	
	\begin{itemize}
		\item 
		\begin{enumerate}
			\item x: $\mathbb{R}^{m X n} = 0.0 $, this is initialization. 
			This is necessary as all the methods will be using the same state 
			variable.
			
			\item x: $\mathbb{R}^{m X n}$ obtained by solving for $x$ using 
			the equation below.
		\begin{equation*}
		\begin{bmatrix}
		1          & t_0       & t_0 ^{2}         & t_0 ^{3}         \\
		\vdots     & \vdots    & \vdots           & \vdots            \\
		1          & t_n       & t_n ^{2}         & t_n ^{3}          \\
		0          & 1         & 2 t_0            & 3 t_0 ^{2}         \\
		\vdots     & \vdots    & \vdots           & \vdots            \\
		0          & 1         & 2 t_n            & 3 t_n ^{2}          \\
		0          & 0         & 2                & 6 t_0              \\
		\vdots     & \vdots    & \vdots           & \vdots            \\
		0          & 0         & 2                & 6 t_n           \\
		0          & 0         & 0                & 6                 \\
		\vdots     & \vdots    & \vdots           & \vdots            \\
		0          & 0         & 0                & 6               \\
		\end{bmatrix}
		\begin{bmatrix}
		x_1  \\
		\vdots \\
		\vdots \\
		\vdots \\
		\vdots \\
		\vdots \\
		\vdots \\
		\vdots \\
		\vdots \\
		\vdots \\
		
		x_{n(m+1)} \\
		\end{bmatrix} = 
		\begin{bmatrix}
		y_0  \\
		\vdots \\
		y_0 ^{(1)} \\
		y_1 ^{(1)} \\
		\vdots \\
		y_n ^{(1)} \\
		y_1 ^{(2)} \\
		\vdots \\
		y_n ^{(2)} \\  
		y_1 ^{(3)} \\
		\vdots \\
		y_n ^{(3)} \\ 
		\end{bmatrix}
		\end{equation*}\\ 
		
		\end{enumerate}
		\item interval: $\mathbb{R}^{k} = t[0] $ to $t[n-2]$, by default we 
		make each point a breakpoint and hence there is a piecewise polynomial 
		between [$t_{i-1},t_i$).
	\end{itemize}	
	
	\item output: NA.	
	
\item exception: NA 
\end{itemize}



\noindent \textbf{evalHermiteCubic($t$,$y$):}
\begin{itemize}
	\item transition: 
		
	\item output:  \textit{out} := $yNew$
	
	$\forall t_i \in t$, yNew.append($y_0$) where $y_0$ is obtained as shown 
	below.
	
	\begin{equation*}
	y_0 = x_{k0} + x_{k1} t + x_{k2} t^{2} + x_{k3} t^{3}\ \ for\ k = 0\ to\ n-2
	\end{equation*}
	
	\item exception: NA
\end{itemize}



%%%%%%%%%%%%%%%%%%%%%%%%%%%%%%%%%%%%%%%%%%%%%%%%%%%%%%%%%%%%%%%%%%%%%%%%%


\noindent \textbf{interpBSpline($t$,$y$):}
\begin{itemize}
	\item transition: 
	
		\begin{itemize}
		\item 
		\begin{enumerate}
			\item x: $\mathbb{R}^{m X n} = 0.0 $, this is initialization. 
			This is necessary as all the methods will be using the same state 
			variable.
			
			\item x: $\mathbb{R}^{m X n}$ obtained by using the formula below.
			
			For $k>0$ to $n$, where $n$ is the number of data points
			\begin{equation*}
			x = v_i ^k = \frac{t-t_i}{t_{i+k} - t_i}
			\end{equation*}
			
			\ms{BSpline gives the v values which will be used recursively in 
			formula defined in IM\ref{IM_BSpline} of the CA document}
			
		\end{enumerate}
		\item interval: $\mathbb{R}^{k} = t[0] $ to $t[n-2]$, by default we 
		make each point a breakpoint and hence there is a piecewise polynomial 
		between [$t_{i-1},t_i$).
	\end{itemize}
	
	
	 
	\item output: NA 	
	 
	\item exception: NA 
\end{itemize}


\noindent \textbf{evalBSpline($t$,$y$):}
\begin{itemize}
	\item transition: 
	
	\item output:  \textit{out} := $yNew$
	
	$\forall t_i \in t$, yNew.append($y_0$) where $y_0$ is obtained as shown 
	below.
	
	\begin{equation*}
	y_0 = B_i ^k (t) = v_i ^k (t) B_i ^{k-1} (t) + (1 - v_{i+1} ^k (t))B_{i+1} 
	^{k-1} 
	(t)
	\end{equation*} 
	for $k>0$.\\
	\ms{I want the notation of `B' so that it is easy to understand the 
	coefficients, thats why I have $y_0 = B...$}
	\item exception: NA
\end{itemize}


\subsubsection{Local Functions}

NA

%%%%%%%%%%%%%%%%%%%%%%%%%%%%%%%%%%%%%%%%%%%%%%%%%%%%%%%%%%%%%%%%%5

\section{MIS of Regression module} \label{mReg}

This module corresponds to R\ref{R_Int_or_Reg}, R\ref{R_Rmethod} and 
R\ref{R_ICalculate} in section \ref{FReq} of CA document.

\subsection{Module}

reg

\subsection{Uses}

Input(\ref{mInput}), Output(\ref{mOutput}), Sequence services(\ref{seqSer})

\subsection{Syntax}

\subsubsection{Exported Constants}

NA

\subsubsection{Exported Access Programs}

\begin{center}
	\begin{tabular}{p{4cm} p{4cm} p{4cm} p{1cm}}
		\hline
		\textbf{Name} & \textbf{In} & \textbf{Out} & \textbf{Exceptions} \\
		\hline
		regNormalEq & t:$\mathbb{R}^n$,y:$\mathbb{R}^n$, deg:$\mathbb{N}$  & 
		- & - \\
		regAugSys & t:$\mathbb{R}^n$,y:$\mathbb{R}^n$, deg:$\mathbb{N}$ & -
		& - \\
		regOrthogonalTn & t:$\mathbb{R}^n$,y:$\mathbb{R}^n$,  
		deg:$\mathbb{N}$ & -
		& - \\
		evalReg & t:$\mathbb{R}^n$ & yNew:$\mathbb{R}^n$ & - \\
		
		\hline
	\end{tabular}
\end{center}

\wss{Again, it is unclear why you have an input parameters module with state
  information if you aren't going to use that state information.}

\subsection{Semantics}

\subsubsection{State Variables}
\begin{itemize}

\item x: $\mathbb{R}^m$\\
\end{itemize}

\wss{You should give a clue on how to interpret this state variable.  Is it the
  coefficients for the polynomial?}

\subsubsection{Environment Variables}

NA

\subsubsection{Assumptions}
\begin{itemize}
\item input.verifyDegree() will be called before any of the access programs 
will be used.
\item evalReg() will be called after calling (regNormalEq() $\lor$  
regAugSys() $\lor$ regOrthogonalTn()).
\end{itemize}
\subsubsection{Access Routine Semantics}

\noindent reg.$x$:
\begin{itemize}
	\item transition: NA	
	\item output: \textit{out} := $x$
	\item exception: NA
\end{itemize}

~\newline
~\newline
\noindent \textbf{regNormalEq($t$,$y$,$deg$):}
\begin{itemize}
	\item transition: 
	\begin{itemize}
		\item $degVerify$ = verifyDegree($deg$)
		\item ($degVerify == True) \Longrightarrow  x ^ n = 0.0$ \ms{This is 
		initialization with maximum length possible. For a data of length n, 
		you can maximum get n coefficients in the polynomial.}
		\item $x ^ m = \mathbb{R}^m$, obtained by solving for $x$ in the 
		equation below.\ms{m is the degree of the coefficients of the 
		polynomial.}
		\begin{equation*}
		A^{T} Ax = A^{T}y
		\end{equation*}  
	\end{itemize} 
	\item output: NA
	\item exception: NA
\end{itemize}

~\newline
~\newline

\noindent \textbf{regAugSys($t$,$y$,$deg$):}
\begin{itemize}
\item transition: 
\begin{itemize}
	\item $degVerify$ = verifyDegree($deg$)
	\item ($degVerify == True) \Longrightarrow  x ^ n = 0.0$ \ms{This is 
		initialization with maximum length possible. For a data of length n, 
		you can maximum get n coefficients in the polynomial.}
	\item $x ^ m = \mathbb{R}^m$, obtained by solving for $x$ in the 
	equation below.\ms{m is the degree of the coefficients of the 
		polynomial.}
		\begin{equation*}
		r + Ax = y
		\end{equation*}
		\begin{equation*}
		A^{T} r = 0
		\end{equation*}
		\text{Where $r$ is the residual vector.}
	\end{itemize}	
	\item output: NA 
		
	\item exception: NA
\end{itemize}

~\newline
~\newline

\noindent \textbf{regOrthogonalTn($t$,$y$,$deg$):}
\begin{itemize}
\item transition: 
\begin{itemize}
	\item $degVerify$ = verifyDegree($deg$)
	\item ($degVerify == True) \Longrightarrow  x ^ n = 0.0$ \ms{This is 
		initialization with maximum length possible. For a data of length n, 
		you can maximum get n coefficients in the polynomial.}
	\item $x ^ m = \mathbb{R}^m$, obtained by solving for $x$ in the 
	equation below.\ms{m is the degree of the coefficients of the 
		polynomial.}
		\begin{equation*}
		Ax = Py
		\end{equation*}\\
		\text{Where $P$ is the orthogonality matrix.}
	\end{itemize}	
	\item output: NA
	
	
	\item exception: NA
\end{itemize}

\noindent \textbf{evalReg($t$):}
\begin{itemize}
	\item transition: NA
	\item output: \textit{out} := $yNew$, where
	$ \forall\ t_i\ \in\ t$,  
	\begin{equation*}
	yNew.append(np.poly1d(x)(t_i))
	\end{equation*}
	\item exception: NA
\end{itemize}

\wss{Is this connected to the state variable? I think so?}  \wss{You should use
  calls to Python in your specification.  Say things mathematically.}

\subsubsection{Local Functions}

NA

%%%%%%%%%%%%%%%%%%%%%%%%%%%%%%%%%%%%%%%%%%%%%%%%%%%%%%%%%%%%%%%%%%%%%%%%%%%%%

\section{MIS of Output module} \label{mOutput}

This module corresponds to R\ref{R_IOutput} in the CA document. The secrets of 
this module are how the output is given to the user program.

\subsection{Module}

output

\subsection{Uses}

Plot(\ref{plot}), Sequence services(\ref{seqSer})

\subsection{Syntax}

\begin{tabular}{p{3cm} p{3cm} p{4cm} >{\raggedright\arraybackslash}p{5cm}}
	\toprule
	\textbf{Name} & \textbf{In} & \textbf{Out} & \textbf{Exceptions} \\
	\midrule
	
	plotDataFit & t:$\mathbb{R}^n$,y:$\mathbb{R}^n$ &  & -  \\
	
	coeffPlotScreen & t:$\mathbb{R}^n$,y:$\mathbb{R}^n$,x:$\mathbb{R}^m$ & 
	x:$\mathbb{R}^m$, plot &  -\\
	
	coeffFile& x:$\mathbb{R}^m$,t:$\mathbb{R}^n$,y:$\mathbb{R}^n$ & 
	x:$\mathbb{R}^m$
	& - \\
	
	\bottomrule
\end{tabular}


\subsubsection{Exported Constants}
NA

\subsubsection{Exported Access Programs}
\begin{tabular}{p{3cm} p{3cm} p{4cm} >{\raggedright\arraybackslash}p{5cm}}
	\toprule
	\textbf{Name} & \textbf{In} & \textbf{Out} & \textbf{Exceptions} \\
	\midrule
	
	plotDataFit & t:$\mathbb{R}^n$,y:$\mathbb{R}^n$ &  & -  \\
	
	coeffPlotScreen & t:$\mathbb{R}^n$,y:$\mathbb{R}^n$,x:$\mathbb{R}^m$ & 
	x:$\mathbb{R}^m$, plot &  -\\
	
	coeffFile& x:$\mathbb{R}^m$,t:$\mathbb{R}^n$,y:$\mathbb{R}^n$ & 
	x:$\mathbb{R}^m$
	& - \\
		
	\bottomrule
\end{tabular}


\subsection{Semantics}

\subsubsection{State Variables}
NA


\subsubsection{Environment Variables}


OutputFile: Results.csv

\subsubsection{Assumptions}

\begin{itemize}
	
	\item The interpolation or the regression module will be called before the 
	output module.
	
\end{itemize}
\subsubsection{Access Routine Semantics}



\noindent \textbf{plotDataFit($t$,$y$):}
\begin{itemize}
	\item transition: 
	\begin{itemize}
		\item Call the appropriate evaluate method from the interpolation or 
		regression module as shown below..
		
		\begin{enumerate}
			
			\item $yNew = $\ interp.evalMonomial($t$,$y$) or
			interp.evalLagrange($t$,$y$) or
			interp.evalNewton($t$,$y$) or
			interp.evalHermiteCubic($t$,$y$) or
			interp.evalBSpline($t$,$y$) or
			reg.evalReg($t$)
			
			\item plot($t,y$) and plot($t$,$yNew$)
\end{enumerate}	
	\end{itemize}

	\item output: \text{out:=} plot
	
	\item exception: NA
\end{itemize}

\noindent \textbf{coeffPlotScreen($t$,$y$):}
\begin{itemize}
	\item transition :
	\begin{itemize}
		\item plotDataFit($t$,$y$)
	\end{itemize}
	\item output: \text{Out:= $interp.x\ \text{or}\ reg.x$, plot  } 
	\item exception: NA
\end{itemize}


~\newline
\noindent \textbf{coeffFile():}
\begin{itemize}
	\item transition: 
	\begin{itemize}
		
		\item Create a csv file named `Result'
		\item write $interp.x\ \text{or}\ reg.x$
	\end{itemize}
	\item output: NA
	\item exception: NA 
\end{itemize}

\subsubsection{Local Functions}

NA


%####################################################################

\section{MIS of Sequence Services} \label{seqSer} 

\subsection{Module}

seqSer \ms{Provided by numpy Python}

\subsection{Uses} 
NA
\subsection{Syntax}

\subsubsection{Exported Constants}
NA
\subsubsection{Exported Access Programs}

\begin{center}
	\begin{tabular}{p{2cm} p{4cm} p{4cm} p{2cm}}
		\hline
		\textbf{Name} & \textbf{In} & \textbf{Out} & \textbf{Exceptions} \\
		\hline
		isAscending & $arr: \mathbb{R}^n$ & $\mathbb{B}$ & -\\
		unique & $arr: \mathbb{R}^n$ & $uniArr: \mathbb{R}^m$ & -\\
		array\_equal & $arr1: \mathbb{R}^n$,$arr2: \mathbb{R}^n$ & $\mathbb{B}$ 
		& - \\ 
		\hline
	\end{tabular}
\end{center}

\subsection{Semantics}

\subsubsection{State Variables}

NA

\subsubsection{Environment Variables}

NA
\subsubsection{Assumptions}

NA

\subsubsection{Access Routine Semantics}

\noindent unique($arr$):
\begin{itemize}
	\item transition: NA
	\item output: $out := uniArr: \mathbb{R}^m$ where $uniArr$ is given by,
	\begin{enumerate}
	\item $arr[i]\neq arr[j],\forall i,j \in [(0\ \text{to}\ (|arr|-1)] \land 
	j \neq i \Longrightarrow\\ \forall i\in[0\ \text{to}\ |arr|-1], uniArr[i] 
	\coloneqq arr[i]$
	\end{enumerate}
	
	\item exception: NA
\end{itemize}


\noindent isAscending($arr$)
\begin{itemize}
	\item transition:NA
	\item output: $out := \neg \exists(i | i \in [0..|arr|-2] : arr_{i+1} < 
	arr_i)$
	\item exception: NA
\end{itemize}



\noindent array\_equal($arr1$,$arr2$):
\begin{itemize}
	\item transition: NA
	\item output: $out := (|arr1| == |arr2|) \land (\forall i \in [0\ 
	\text{to}\ |arr1|-1], arr1[i] == arr2[i]) \Longrightarrow True $
	
	\item exception: NA
\end{itemize}

\subsubsection{Local Functions}

NA

\subsubsection{Considerations}
This module is provided by numpy in Python.


\section{MIS of Plot Module} \label{plot}

\subsection{Module}

plot

\subsection{Uses}

NA

\subsection{Syntax}

\subsubsection{Exported Access Programs}

\begin{center}
	\begin{tabular}{p{2cm} p{8cm} p{2cm} p{2cm}}
		\hline
		\textbf{Name} & \textbf{In} & \textbf{Out} & \textbf{Exceptions} \\
		\hline
		Plot & $X: \mathbb{R}^n, Y: \mathbb{R}^n$ & ~ & SeqSizeMismatch\\
		\hline
	\end{tabular}
\end{center}

\subsection{Semantics}

\subsubsection{State Variables}

NA

\subsubsection{Environment Variables}

win: 2D sequence of pixels displayed on the screen\\ \wss{good}

\subsubsection{Assumptions}

NA

\subsubsection{Access Routine Semantics}

\noindent Plot($X, Y$)
\begin{itemize}
	\item transition: modify win so that it displays an x-y graph of the data 
	points showing X and the corresponding Y values.  X is the independent 
	variable and Y as the dependent variable.
	\item exception: $( |X| \neq |Y| \Rightarrow \mbox{SeqSizeMismatch})$
\end{itemize}

\newpage




\bibliographystyle {plainnat}
\bibliography {../../../ReferenceMaterial/References}

\newpage



%\section{Appendix} \label{Appendix}

%\wss{Extra information if required}

\end{document}