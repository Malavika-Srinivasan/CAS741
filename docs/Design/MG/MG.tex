\documentclass[12pt, titlepage]{article}

\usepackage{fullpage}
\usepackage[round]{natbib}
\usepackage{multirow}
\usepackage{booktabs}
\usepackage{tabularx}
\usepackage{graphicx}
\usepackage{float}
\usepackage{hyperref}
\hypersetup{
    colorlinks,
    citecolor=black,
    filecolor=black,
    linkcolor=red,
    urlcolor=blue
}

\input{../../Comments}

\newcounter{acnum}
\newcommand{\actheacnum}{AC\theacnum}
\newcommand{\acref}[1]{AC\ref{#1}}

\newcounter{ucnum}
\newcommand{\uctheucnum}{UC\theucnum}
\newcommand{\uref}[1]{UC\ref{#1}}

\newcounter{mnum}
\newcommand{\mthemnum}{M\themnum}
\newcommand{\mref}[1]{M\ref{#1}}

\newcommand{\famname}{CFS} % PUT YOUR PROGRAM NAME HERE


\begin{document}

\title{Module Guide: Curve Fitting Software} 
\author{Malavika Srinivasan}
\date{\today}

\maketitle

\pagenumbering{roman}

\section{Revision History}

\begin{tabularx}{\textwidth}{p{3cm}p{2cm}X}
\toprule {\bf Date} & {\bf Version} & {\bf Notes}\\
\midrule
Oct 27, 2018 & 1.0 & First draft by Malavika\\
Oct 30, 2018 & 1.1 & Second draft by Malavika\\
Oct 27, 2018 & 1.2 & Third draft by Malavika\\
Nov 5, 2018 & 1.3 & Fourth draft by Malavika\\
Dec 24, 2018 & 1.4 & Final draft by Malavika\\
\bottomrule
\end{tabularx}

\newpage

\tableofcontents

\listoftables

\listoffigures

\newpage

\pagenumbering{arabic}

\section{Introduction}

Decomposing a system into modules is a commonly accepted approach to developing
software.  A module is a work assignment for a programmer or programming
team~\citep{ParnasEtAl1984}.  We advocate a decomposition
based on the principle of information hiding~\citep{Parnas1972a}.  This
principle supports design for change, because the ``secrets'' that each module
hides represent likely future changes.  Design for change is valuable in SC,
where modifications are frequent, especially during initial development as the
solution space is explored.  

Our design follows the rules layed out by \citet{ParnasEtAl1984}, as follows:
\begin{itemize}
\item System details that are likely to change independently should be the
  secrets of separate modules.
\item Each data structure is used in only one module.
\item Any other program that requires information stored in a module's data
  structures must obtain it by calling access programs belonging to that module.
\end{itemize}

After completing the first stage of the design, the Software Requirements
Specification (SRS), the Module Guide (MG) is developed~\citep{ParnasEtAl1984}. The MG
specifies the modular structure of the system and is intended to allow both
designers and maintainers to easily identify the parts of the software.  The
potential readers of this document are as follows:

\begin{itemize}
\item New project members: This document can be a guide for a new project member
  to easily understand the overall structure and quickly find the
  relevant modules they are searching for.
\item Maintainers: The hierarchical structure of the module guide improves the
  maintainers' understanding when they need to make changes to the system. It is
  important for a maintainer to update the relevant sections of the document
  after changes have been made.
\item Designers: Once the module guide has been written, it can be used to
  check for consistency, feasibility and flexibility. Designers can verify the
  system in various ways, such as consistency among modules, feasibility of the
  decomposition, and flexibility of the design.
\end{itemize}

The rest of the document is organized as follows. Section
\ref{SecChange} lists the anticipated and unlikely changes of the software
requirements. Section \ref{SecMH} summarizes the module decomposition that
was constructed according to the likely changes. Section \ref{SecConnection}
specifies the connections between the software requirements and the
modules. Section \ref{SecMD} gives a detailed description of the
modules. Section \ref{SecTM} includes two traceability matrices. One checks
the completeness of the design against the requirements provided in the SRS. The
other shows the relation between anticipated changes and the modules. Section
\ref{SecUse} describes the use relation between modules.

\section{Anticipated and Unlikely Changes} \label{SecChange}

This section lists possible changes to the system. According to the likeliness
of the change, the possible changes are classified into two
categories. Anticipated changes are listed in Section \ref{SecAchange}, and
unlikely changes are listed in Section \ref{SecUchange}.

\subsection{Anticipated Changes} \label{SecAchange}

Anticipated changes are the source of the information that is to be hidden
inside the modules. Ideally, changing one of the anticipated changes will only
require changing the one module that hides the associated decision. The approach
adapted here is called design for
change.

\begin{description}
\item[\refstepcounter{acnum} \actheacnum \label{acHardware}:] The specific
  hardware on which the software is running. 
\item[\refstepcounter{acnum} \actheacnum \label{acOutput}:] The format of output
  data. Instead of best fit coefficients and plot, they may want to obtain the
  value of `y' at a particular `t' or a series of `t' values where $(t_i,y_i)$
  is the input data at $i = 0,1,2,...n$.
\item[\refstepcounter{acnum} \actheacnum \label{acDegree}:] The user may not
  want to input the degree of the polynomial for regression and my want it to be
  computed by \famname{}.
\item[\refstepcounter{acnum} \actheacnum \label{acNewIntReg}:] Additional 
methods of interpolation and regression than the ones covered in this software 
may be added in the future.
\end{description}

\subsection{Unlikely Changes} \label{SecUchange}

The module design should be as general as possible. However, a general system is
more complex. Sometimes this complexity is not necessary. Fixing some design
decisions at the system architecture stage can simplify the software design. If
these decision should later need to be changed, then many parts of the design
will potentially need to be modified. Hence, it is not intended that these
decisions will be changed.

\begin{description}
\item[\refstepcounter{ucnum} \uctheucnum \label{ucIO}:] Input/Output devices
  (Input: File and/or Keyboard, Output: File, Memory, and/or Screen).
\item[\refstepcounter{ucnum} \uctheucnum \label{ucNonLinear}:] If \famname{} is
  used with non linear systems, then the results from the software are no longer
  valid.
\end{description}

\section{Module Hierarchy} \label{SecMH}

This section provides an overview of the module design. Modules are summarized
in a hierarchy decomposed by secrets in Table \ref{TblMH}. The modules listed
below, which are leaves in the hierarchy tree, are the modules that will
actually be implemented.

\begin{description}
\item [\refstepcounter{mnum} \mthemnum \label{mHH}:] Hardware-Hiding Module
\item [\refstepcounter{mnum} \mthemnum \label{mBH}:] Behaviour-Hiding Module
\item [\refstepcounter{mnum} \mthemnum \label{mControl}:] User-program Module
\item [\refstepcounter{mnum} \mthemnum \label{mInput}:] Input Module
\item [\refstepcounter{mnum} \mthemnum \label{mOutput}:] Output Module
\item [\refstepcounter{mnum} \mthemnum \label{mSD}:] Software Decision Module
\item [\refstepcounter{mnum} \mthemnum \label{mSeqSer}:] Sequence Services Module 
\item [\refstepcounter{mnum} \mthemnum \label{mInterp}:] Interpolation Module
\item [\refstepcounter{mnum} \mthemnum \label{mLinearReg}:] Regression Module
\item [\refstepcounter{mnum} \mthemnum \label{mPlot}:] Plot Module
\end{description}

Note that \mref{mHH} is a commonly used module and is already implemented by the
operating system.  It will not be reimplemented.  Similarly, \mref{mSeqSer} and
\mref{mPlot} are already available in Python and will not be reimplemented.

\begin{table}[h!]
\centering
\begin{tabular}{p{0.3\textwidth} p{0.6\textwidth}}
\toprule
\textbf{Level 1} & \textbf{Level 2}\\
\midrule

{Hardware-Hiding Module} & ~ \\
\midrule

\multirow{3}{0.3\textwidth}{Behaviour-Hiding Module}& Input\\
& Output\\

\midrule

\multirow{4}{0.3\textwidth}{Software Decision Module} & Sequence Services\\
& Interpolation\\
& Regression\\
& Plot\\
\bottomrule

\end{tabular}
\caption{Module Hierarchy}
\label{TblMH}
\end{table}

\section{Connection Between Requirements and Design} \label{SecConnection}

The design of the system is intended to satisfy the requirements developed in
the SRS. In this stage, the system is decomposed into modules. The connection
between requirements and modules is listed in Table \ref{TblRT}.

\section{Module Decomposition} \label{SecMD}

Modules are decomposed according to the principle of ``information hiding''
proposed by \citet{ParnasEtAl1984}. The \emph{Secrets} field in a module
decomposition is a brief statement of the design decision hidden by the
module. The \emph{Services} field specifies \emph{what} the module will do
without documenting \emph{how} to do it. For each module, a suggestion for the
implementing software is given under the \emph{Implemented By} title. If the
entry is \emph{OS}, this means that the module is provided by the operating
system or by standard programming language libraries.  Also indicate if the
module will be implemented specifically for the software.

Only the leaf modules in the hierarchy have to be implemented. If a dash
(\emph{--}) is shown, this means that the module is not a leaf and will not have
to be implemented. Whether or not this module is implemented depends on the
programming language selected.

\subsection{Hardware Hiding Modules (\mref{mHH})}

\begin{description}
\item[Secrets:]The data structure and algorithm used to implement the virtual
  hardware.
\item[Services:]Serves as a virtual hardware used by the rest of the
  system. This module provides the interface between the hardware and the
  software. So, the system can use it to display outputs or to accept inputs.
\item[Implemented By:] OS
\end{description}

\subsection{Behaviour-Hiding Module (\mref{mBH})}

\begin{description}
\item[Secrets:]The contents of the required behaviours.
\item[Services:]Includes programs that provide externally visible behaviour of
  the system as specified in the software requirements specification (SRS)
  documents. This module serves as a communication layer between the
  hardware-hiding module and the software decision module. The programs in this
  module will need to change if there are changes in the SRS.
\item[Implemented By:] --
\end{description}
%%%%%%%%%%%%%%%%%%%%%%%%%%%%%%%%%%%%%%%%%%%%%%%%%%%%%%%%%%%%
\subsubsection{User-program Module (\mref{mControl})}

\begin{description}
\item[Secrets:]This is the user program which
  uses the Interpolation, Regression and Output modules to use the appropriate 
  service. This is not a module of \famname{} but is an external program which 
  is included here for ease of understanding.
   \wss{This isn't
    actually a module.  If you want to include it, you should say this.  This is
    an external program that uses the services provided by the other modules.}
\ms{Mentioned details about UserProgram.}
	\item[Services:] NA.
	\item[Implemented By:] User
\end{description}

%%%%%%%%%%%%%%%%%%%%%%%%%%%%%%%%%%%%%%%%%%%%%%%%%%%%%%%%%%%%
\subsubsection{Input Module (\mref{mInput})}

\begin{description}
	\item[Secrets:]The constraints of input data.
	\item[Services:]Verifies the input data. 
		\begin{enumerate}
			\item A `TypeError' exception is raised if there is a type mismatch.
			\item A 'LengthError' exception is raised if length of input arrays 
			is equal to 1.
			\item A 'LengthMismatchError' exception is raised if length of the 
			two input arrays are not the same.
			\item A 'valueError' exception is raised if the degree input is not 
			a natural number.
		\end{enumerate} 
	\item[Implemented By:] \famname{}
\end{description}
%%%%%%%%%%%%%%%%%%%%%%%%%%%%%%%%%%%%%%%%%%%%%%%%%%%%%%%%%%%%



\subsubsection{Output Module (\mref{mOutput})}

\begin{description}
	\item[Secrets:] The format and structure of the output data.
	\item[Services:] Outputs the best fit parameters and plots the output
          along with input data and best fit parameters.
	\item[Implemented By:] \famname{}
\end{description} 

%%%%%%%%%%%%%%%%%%%%%%%%%%%%%%%%%%%%%%%%%%%%%%%%%%%%%%%%%%%%


\subsection{Software Decision Module (\mref{mSD})}

\begin{description}
\item[Secrets:] The design decision based on mathematical theorems, physical
  facts, or programming considerations. The secrets of this module are
  \emph{not} described in the CA document.
\item[Services:] Includes data structure and algorithms used in the system that
  do not provide direct interaction with the user. 
\item[Implemented By:] --
\end{description}


\subsubsection{Sequence Services Module  (\mref{mSeqSer})}

\begin{description}
	\item[Secrets:] The data structure for a sequence data type.
	\item[Services:] Provides array manipulation, including building an array,
	accessing a specific entry etc.
	\item[Implemented By:] Python - Numpy
\end{description} 

\subsubsection{Interpolation Module (\mref{mInterp})}

\begin{description}
	\item[Secrets:] The algorithms used for interpolation.
	\item[Services:] Provides the fit coefficients. \wss{I removed the word
            best.  With interpolation there is only one fit.}\ms{Thank you :)}
	\item[Implemented By:] \famname{}
\end{description}

\subsubsection{Linear Regression (\mref{mLinearReg})}

\begin{description}
	\item[Secrets:] The algorithm used for linear regression.
	\item[Services:] Provides the best fit coefficients.
	\item[Implemented By:] \famname{}
\end{description}

\subsubsection{Plot Module (\mref{mPlot})}

\begin{description}
	\item[Secrets:] The data structures and algorithms for plotting data graphically.
	\item[Services:] Provides a plot function.
	\item[Implemented By:] Python - Matplotlib
\end{description}

\section{Traceability Matrix} \label{SecTM}

This section shows two traceability matrices: between the modules and the
requirements and between the modules and the anticipated changes.

\begin{table}[H]
\centering
\begin{tabular}{p{0.2\textwidth} p{0.6\textwidth}}
\toprule
\textbf{Req.} & \textbf{Modules}\\
\midrule
R1 & \mref{mHH},\mref{mBH}, \mref{mControl}, \mref{mInput}, \mref{mSeqSer} \\
R2 & \mref{mBH}, \mref{mControl} \mref{mInput}, \mref{mSeqSer}\\
R3 & \mref{mBH}, \mref{mControl} \mref{mInput}, \mref{mSeqSer}\\
R4 & \mref{mHH},\mref{mBH}, \mref{mControl}\\
R5 & \mref{mHH},\mref{mBH}, \mref{mControl}\\ 
R6 & \mref{mHH},\mref{mBH}, \mref{mControl}\\
R7 & \mref{mHH},\mref{mBH}, \mref{mControl}\\
R8 & \mref{mHH},\mref{mBH}, \mref{mControl}\\
R9 & \mref{mControl},\mref{mSD}, \mref{mSeqSer}, \mref{mInterp}, \mref{mLinearReg}\\
R10 & \mref{mHH},\mref{mBH}, \mref{mOutput}, \mref{mSeqSer}, \mref{mPlot}\\

\bottomrule
\end{tabular}
\caption{Trace Between Requirements and Modules}
\label{TblRT}
\end{table}

\begin{table}[H]
\centering
\begin{tabular}{p{0.2\textwidth} p{0.6\textwidth}}
\toprule
\textbf{AC} & \textbf{Modules}\\
\midrule
\acref{acHardware} & \mref{mHH}\\
\acref{acOutput} & \mref{mOutput}\\
\acref{acDegree} & \mref{mLinearReg}\\
\acref{acNewIntReg} & \mref{mInterp}, \mref{mLinearReg}\\
\bottomrule
\end{tabular}
\caption{Trace Between Anticipated Changes and Modules}
\label{TblACT}
\end{table}

\section{Use Hierarchy Between Modules} \label{SecUse}

In this section, the uses hierarchy between modules is
provided. \citet{Parnas1978} said of two programs A and B that A {\em uses} B if
correct execution of B may be necessary for A to complete the task described in
its specification. That is, A {\em uses} B if there exist situations in which
the correct functioning of A depends upon the availability of a correct
implementation of B.  Figure \ref{FigUH} illustrates the use relation between
the modules. It can be seen that the graph is a directed acyclic graph
(DAG). Each level of the hierarchy offers a testable and usable subset of the
system, and modules in the higher level of the hierarchy are essentially simpler
because they use modules from the lower levels.

\begin{figure}[H]
\centering
\includegraphics[width=0.8\textwidth]{UsesHierarchy.jpg}
\caption{Use hierarchy among modules}
\label{FigUH}
\end{figure}



%\section*{References}

\bibliographystyle {plainnat}
\bibliography{../../../ReferenceMaterial/References}

\end{document}