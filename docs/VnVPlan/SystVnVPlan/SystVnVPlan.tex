\documentclass[12pt, titlepage]{article}
\newcommand{\famname}{CFS} % PUT YOUR PROGRAM NAME HERE
\usepackage{booktabs}
\usepackage{tabularx}
\usepackage{hyperref}
\hypersetup{
    colorlinks,
    citecolor=black,
    filecolor=black,
    linkcolor=red,
    urlcolor=blue
}
\usepackage[round]{natbib}
\usepackage{xcolor}
\usepackage{amsmath, mathtools}
\usepackage{amsfonts}
\usepackage{amssymb}
\usepackage{graphicx}
\usepackage{colortbl}
\usepackage{xr}
\usepackage{hyperref}
\usepackage{longtable}
\usepackage{xfrac}
\usepackage{float}
\usepackage{siunitx}
\usepackage{caption}
\usepackage{pdflscape}
\usepackage{afterpage}
\usepackage{comment}
\input{../../Comments}


\usepackage{titlesec}


\begin{document}

\title{CFS: System Verification and Validation Plan} 
\author{Malavika Srinivasan}
\date{\today}
	
\maketitle

\pagenumbering{roman}


\section{Revision History}

\begin{tabularx}{\textwidth}{p{3cm}p{2cm}X}
\toprule {\bf Date} & {\bf Version} & {\bf Notes}\\
\midrule
Oct 16, 2018 & 1.0 & First draft by Malavika\\
Oct 23, 2018 & 2.0 & Second draft by Malavika\\

\bottomrule
\end{tabularx}

~\newpage

\section{Symbols, Abbreviations and Acronyms}

\renewcommand{\arraystretch}{1.2}
\begin{tabular}{l l} 
  \toprule		
  \textbf{symbol} & \textbf{description}\\
  \midrule 
  T & Test\\
  \bottomrule
\end{tabular}\\
\\
Also see the table of symbols in CA at: \url{https://github.com/Malavika-Srinivasan/CAS741/tree/master/docs/SRS/CA.pdf}\\


\newpage

\tableofcontents

\listoftables

\listoffigures

\newpage

\pagenumbering{arabic}


This document explains the verification and validation plan to improve the quality of \famname{}. It is organized into different sections which gives a detailed description of the \famname{} along with its goals and objectives, qualities which are important for \famname{} and test cases for the functional and non functional requirements mentioned in the CA document.



\section{General Information}

This section explains the summary of what is being tested in this document, the objectives of this document and references for this document. 


\subsection{Summary}

This document will summarize the plan for verification and validation of \famname{} in compliance with the requirements specified in the CA document found at found at \url{https://github.com/Malavika-Srinivasan/CAS741/tree/master/docs/SRS/CA.pdf} 
The goal statement as found in the CA document is presented below.\\
\noindent ``Given the set of data points, the choice of software from \famname{} and the variabilities of the software the \famname{} should:

\begin{enumerate}
	
	\item Compute the parameters of the curve which is the best possible fit through the set of data points.	''
\end{enumerate}


\subsection{Objectives}

The goal of verification and validation is to improve the quality of the \famname{} and obtain confidence in the software implementation. There are several standards which define software quality. According to the quality model of ISO 9126,  software quality is described as a structured set of characteristics namely - Functional suitability, Performance, efficiency, Compatibility, Usability,  Reliability, Security, Maintainability and Portability (~\cite{ISO9126}). The qualities which are important concerning \famname{} are correctness(functional suitability), maintainability, re-usability and portability. The definitions of the above mentioned qualities are explained below.

\subsubsection {Functional Suitability}

This characteristic represents the degree to which a product or system provides
functions that meet stated and implied needs when used under specified
conditions. It can be further characterized into completeness, correctness and
appropriateness. In this project, we focus only on correctness which is defined as shown below.
\paragraph{Correctness }
The degree to which a product or system provides the correct results with the
needed degree of precision.

\subsubsection{Maintainability}
This characteristic represents the degree of effectiveness and efficiency with
which a product or system can be modified to improve it, correct it or adapt it
to changes in the environment, and in requirements. 

\subsubsection{Portability}
The degree of effectiveness and efficiency with which a system, product or
component can be transferred from one hardware, software or other operational
or usage environment to another. 


\subsection{References}

Throughout this document, we refer to the terminologies that have been already explained in the CA document for Program \famname{}.

\section{Plan}
	
\subsection{Verification and Validation Team}

\begin{enumerate}
	\item Malavika srinivasan
\end{enumerate}


\subsection{SRS Verification Plan}

The CA document for \famname{} will be reviewed by Dr.Spencer Smith and my classmate Mr.Robert White.

\subsection{Design Verification Plan}

My design will be verified with the help of my supervisor Dr.Spencer Smith and my classmates Ms.Jennifer Garner and Mr.Brooks MacLachlan.

\subsection{Implementation Verification Plan}


My implementation will be verified by the tests listed in this document and the unitVnVplan document. My classmate Ms.Vajiheh Motamer will help me verify the document.	

\subsection{Software Validation Plan}

Not Applicable

\section{System Test Description}

System testing is a process in which we test the overall working of the system. The instance models in CA document will be tested here. This does not test the individual units or modules of the system. It is a black box testing approach.
	
\subsection{Tests for Functional Requirements}

\subsubsection{Input testing}

\begin{enumerate}

\item{\textbf{T1: $I^{st}$ test case for faulty inputs}\\}

Control: Automatic

Initial State: NA

Input: t=[1] , y=[2] (length 1)

Output: Error message (``Please enter atleast 2 points'')

How test will be performed: Pytest




\item{\textbf{T2: ${II}^{nd}$ test case for faulty inputs}\\}

Control: Automatic

Initial State: NA

Input: t=[1,2,3,4,t] , y=[2,2,3,4,5] 

Output: Error message (``Please enter only numbers'')

How test will be performed: Pytest




\item{\textbf{T3: Test case for unordered inputs}\\}

Control: Automatic

Initial State: NA

Input: t=[1,2,5,4] , y=[2,0,3,4] 

Output: Error message (``Wrong input: $t_{i+1}$ must be greater than $t_i$'')

How test will be performed: Pytest



\subsubsection{Interpolation Testing}

%#######################       Monomial    ################################



\item{\textbf{T4: $I^{st}$ test case for monomial interpolation} \\}

Control: Automatic 

Initial State: NA

Input: Data points t = [0,1,2], y = [0,1,2]

Output: [0, 1] (Coefficients of t, starting from $t^{0}$.)

How test will be performed: Pyunit\\




\item{\textbf{T5: ${II}^{nd}$ test case for Monomial interpolation}}\\

Control: Automatic
					
Initial State: NA
					
Input: Data points x = [-2,0,1], y = [-27,-1,0]

Output: [-1,5,-4](Coefficients of t, starting from $t^{0}$.)

How test will be performed: Pytest

Test case reference: Page 314, example 7.1 of ~\cite{Health1997}\\


%#######################       Lagrange    ################################
					
\item{\textbf{T6: $I^{st}$ test case for Lagrange's interpolation} \\}

Control: Automatic 

Initial State: NA

Input: Data points t = [0,1,2], y = [0,1,2]

Output: [0, 1] (Coefficients of t, starting from $t^{0}$.)

How test will be performed: Pyunit\\



\item{\textbf{T7: ${II}^{nd}$ test case for Lagrange's interpolation} \\}


Control: Automatic

Initial State: NA

Input: Data points t = [-2,0,1], y = [-27,-1,0]

Output: [-1,5,-4]

How test will be performed: Pytest

Test case reference: Page 314, example 7.1 of  ~\cite{Health1997}\\

%#######################       Newton    ################################

\item{\textbf{T8: $I^{st}$ test case1 for Newton's interpolation} \\}

Control: Automatic 

Initial State: NA

Input: Data points t = [0,1,2], y = [0,1,2]

Output: [0, 1] (Coefficients of t, starting from $t^{0}$.)

How test will be performed: Pyunit\\


\item{\textbf{T9: ${II}^{nd}$ test case for Newton's interpolation} \\}

Control: Automatic

Initial State: NA

Input: Data points t = [-2,0,1], y = [-27,-1,0]

Output: [-1,5,-4](Coefficients of t, starting from $t^{0}$.)

How test will be performed: Pytest

Test case reference: Page 314, example 7.1 of \cite{Health1997}\\



%#######################       Hermite Cubic     ################

\item{\textbf{T10: $I^{st}$ test case for hermite cubic interpolation }\\}

Control: Automatic 

Initial State: NA

Input: t = [1,3], y = [2,1]

Output: [1,-5.75, 9.5, 2](Coefficients of t, starting from $t^{0}$.)


How test will be performed: Pytest

Test case reference: \url{https://docs.scipy.org/doc/scipy/reference/generated/scipy.interpolate.PchipInterpolator.html#scipy.interpolate.PchipInterpolator}\\


\item{\textbf{T11: ${II}^{nd}$ test case for hermite cubic interpolation} \\}

Control: Automatic 

Initial State: NA

Input: t = [1, 2, 4, 5], y = [2, 1, 4, 3]

Output: 
~\newline [1.0,1.0, 1.38888889, 2.0] in [1,2),(Coefficients of t, starting from $t^{0}$.)

~\newline [4.0, 4.0, 1.0, 1.0] in [2,4) (Coefficients of t, starting from $t^{0}$.)

~\newline[3.0, 3.61111111, 4.0, 4.0] in [4,5) (Coefficients of t, starting from $t^{0}$.)

How test will be performed: Pytest

Test case reference: \url{https://stackoverflow.com/questions/43458414/python-scipy-how-to-get-cubic-spline-equations-from-cubicspline}\\
%#######################################################
i 

\item{\textbf{T12: $I^{st}$ test case for BSpline interpolation} \\}

Control: Automatic 

Initial State: NA

Input: x = [ 0.0, 1.2,  1.9,  3.2,  4.0,  6.5], y = [ 0.0,  2.3,  3.0,  4.3,  2.9,  3.1], s=0, k=4


Output: 
~\newline [-5.62048630e-18, 2.98780300e+00, -5.74472095e-01,  1.46700914e+01,
-1.03253068e+01,  3.10000000e+00,  0.00000000e+00,  0.00000000e+00,
0.00000000e+00,  0.00000000e+00,  0.00000000e+00]

How test will be performed: Pytest

Test case reference: \url{https://stackoverflow.com/questions/45179024/scipy-bspline-fitting-in-python}\\


%#######################################################

\subsubsection{Regression}



\item{\textbf{T13: $I^{st}$ test case for regression using normal equations} \\}

Control: Automatic

Initial State: NA

Input: t = [0,1,2,3,4,5,6,7,8,9,10], y = [0,1,2,3,4,5,6,7,8,9,10], deg = 1 

Output: 0,1(Coefficients of t, starting from $t^{0}$.)

How test will be performed: Pytest\\

%#######################################################

\item{\textbf{T14: ${II}^{nd}$ test case for regression using normal equations}\\}

Control: Automatic

Initial State: NA

Input: Data points t = [1,2,3], y = [1,3,7], deg = 1

Output: -2.33333333333, 3 (Coefficients of t, starting from $t^{0}$.)

How test will be performed: Pytest

Test case reference: \url{http://www4.ncsu.edu/eos/users/w/white/www/white/ma341/lslecture.PDF}\\

%#######################################################

\item{\textbf{T15: ${III}^{rd}$ test case for regression using normal equations }\\}

Control: Automatic

Initial State: NA

Input: t = [0,200,400,600,800], y = [0.0010, 0.0015, 0.0021, 0.0051, 0.0094], degree = 2

Output: 0.00116857142857, -0.00000408571429, 0.00000001785714 (Coefficients of t, starting from $t^{0}$.)

How test will be performed: Pytest

Test case reference: \url{http://www4.ncsu.edu/eos/users/w/white/www/white/ma341/lslecture.PDF}\\


%#######################################################


\item{\textbf{T16: $I^{st}$ test case for regression using augmented systems}\\}

Control: Automatic

Initial State: NA

Input: Data points t = [0,1,2,3,4,5,6,7,8,9,10], y = [0,1,2,3,4,5,6,7,8,9,10], degree = 1 

Output: 0,1 (Coefficients of t, starting from $t^{0}$.)

How test will be performed: Pytest\\

\item{\textbf{T17: ${II}^{nd}$ test case for regression using augmented systems}\\}

Control: Automatic

Initial State: NA

Input: Data points t = [1,-1, -2], y = [3,-5,12] 

Output: -8, 4, 7 (Coefficients of t, starting from $t^{0}$.)

How test will be performed: Pytest

Test case reference: \url{https://math.stackexchange.com/questions/710750/find-a-second-degree-polynomial-that-goes-through-3-points}\\


\item{\textbf{T18: ${II}^{nd}$ test case for regression using orthogonal transformations}\\}

Control: Automatic

Initial State: NA

Input: Data points t = [0,1,2,3,4,5,6,7,8,9,10], y = [0,1,2,3,4,5,6,7,8,9,10] 

Output: 0,1 (Coefficients of t, starting from $t^{0}$.)

How test will be performed: Pytest\\


\item{\textbf{T19: ${II}^{nd}$ test case for regression using orthogonal transformations}\\}

Control: Automatic

Initial State: NA

Input: Data points t = [1,2,3,4], y = [5,3,2,1] , degree = 1

Output: 6,-1.3 (Coefficients of t, starting from $t^{0}$.)

How test will be performed: Pytest

Test case reference: Page $4$ and $5$ of \url{http://www.maths.lse.ac.uk/personal/james/old_ma201/solns8.pdf}\\



\subsection{Tests for Nonfunctional Requirements}



\item{\textbf{T20: Test case for correctness}\\}

Type: Nonfunctional, Manual
					
Initial State: NA
					
Input/Condition: Results from Matlab for T4, T6, T9, T10, T12, T14, T16 and T19 will be manually compared using relative error by the formula below.

\begin{equation*}
err = \frac{val_{CFS} - val_{Matlab}}{val_{CFS}}
\end{equation*}

\begin{equation*}
err < Admissible\_error
\end{equation*}
					
Output/Result: Pass/Fail
					
How test will be performed: Manual
				

				
%#######################################################					
\item{\textbf{T21: Test case for maintainability}\\}

Type: Nonfunctional, Manual
					
Initial State: NA
					
Input: Module guide, Module Interface specification

Steps: 
\begin{enumerate}
	\item Choose a task such as changing the output type of a module
	\item Give the participants MG and MIS
	\item Ask them to find, which module undergoes change.
\end{enumerate}
					
Output: Pass/ Fail
					
How test will be performed: Manual


%#######################################################

\item{\textbf{T22: Test case for Portability}\\}

Type: Nonfunctional, Manual

Initial State: NA

Input/Condition: Try and run \famname{} in Mac, Windows and Linux using virtual machines.

Output/Result: Pass/Fail

How test will be performed: Manual

\end{enumerate}

\subsection{Traceability Between Test Cases and Requirements}

The following table shows the traceability mapping for test cases, instance models and the requirements. 

\begin{table} [H]
	\caption{Requirements Traceability Matrix}
	\label{Table:Table_Traceability}  
	\begin{tabular}{|c|p{5cm}|p{5cm}|}
		\hline	
		\textbf{Test Number} & \textbf{Instance Models} & \textbf{CA Requirements}\\
		\hline 
		T1&         & R1, R3       \\ \hline
		T2&         & R1, R2       \\ \hline
		T3&         & R1, R3       \\ \hline
		
		T4& IM1,IM3 & R4, R5, R6, R9, R10   \\ \hline
		T5& IM1,IM3& R4, R5, R6, R9, R10   \\ \hline
		
		T6& IM1, IM4& R4, R5, R6, R9, R10   \\ \hline
		T7& IM1, IM4& R4, R5, R6, R9, R10   \\ \hline
		
		T8& IM1,IM5& R4, R5, R6, R9, R10   \\ \hline
		T9& IM1, IM5& R4, R5, R6, R9, R10   \\ \hline
		
		
		T10& IM2, IM6& R4, R5, R7, R9, R10     \\ \hline
		T11& IM2, IM6& R4, R5, R7, R9, R10     \\ \hline
		
		
		T12& IM2, IM7& R4, R5, R7, R9, R10     \\ \hline
		
		
		T13& IM8     & R4, R8, R9, R10  \\ \hline
		T14& IM8     & R4, R8, R9, R10  \\ \hline
		T15& IM8     & R4, R8, R9, R10  \\ \hline
		
		
		T16& IM9     & R4, R8, R9, R10    \\ \hline
		T17& IM9     & R4, R8, R9, r10     \\ \hline
		
		T18& IM10    & R4, R8, R9, R10    \\ \hline
		T19& IM10    & R4, R8, R9, R10    \\ \hline
		
		T20&         & R12\\ \hline
		T21&         & R13\\ \hline
		T22&         & R11\\ \hline
		
	\end{tabular}\\
\end{table}




\section{Static Verification Techniques}

Code walkthrough and inspection will be used to verify the implementation.
				
\bibliographystyle {plainnat}
\bibliography {../../../ReferenceMaterial/References}

\newpage

\section{Appendix}



\subsection{Symbolic Parameters}

The definition of the test cases will call for SYMBOLIC\_CONSTANTS.
Their values are defined in this section for easy maintenance.
\begin{itemize}
	\item $Admissible\_error = 1e-1$
\end{itemize}


%\subsection{Usability Survey Questions?}

%\wss{This is a section that would be appropriate for some projects.}


\bibliographystyle {plainnat}
\bibliography {../../../ReferenceMaterial/References}


\end{document}