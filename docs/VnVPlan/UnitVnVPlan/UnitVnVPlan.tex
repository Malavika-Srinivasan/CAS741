\documentclass[12pt, titlepage]{article}
\usepackage{longtable}
\usepackage{xfrac}
\usepackage{float}
\usepackage{caption}
\usepackage{amsmath, mathtools}
\usepackage{comment}
\usepackage{booktabs}
\usepackage{tabularx}
\usepackage{hyperref}
\hypersetup{
    colorlinks,
    citecolor=black,
    filecolor=black,
    linkcolor=red,
    urlcolor=blue
}
%\usepackage{zref-xr}
%\zxrsetup{toltxlabel}
%\zexternaldocument*{../SystVnVPlan/SystVnVPlan}

\usepackage{xr}
%\zxrsetup{toltxlabel}
\externaldocument{../SystVnVPlan/SystVnVPlan}
\usepackage[round]{natbib}
%\externaldocument{../SystVnVPlan/SystVnVPlan}
%% Comments

\usepackage{color}

\newif\ifcomments\commentstrue

\ifcomments
\newcommand{\authornote}[3]{\textcolor{#1}{[#3 ---#2]}}
\newcommand{\todo}[1]{\textcolor{red}{[TODO: #1]}}
\else
\newcommand{\authornote}[3]{}
\newcommand{\todo}[1]{}
\fi

\newcommand{\wss}[1]{\authornote{blue}{SS}{#1}}
\newcommand{\an}[1]{\authornote{magenta}{Malavika}{#1}}

%% Common Parts

\newcommand{\famname}{CFS} % PUT YOUR PROGRAM NAME HERE %Every program
                                % should have a name



\begin{document}

\title{CFS: Unit Verification and Validation Plan for \famname{}} 
\author{Malavika Srinivasan}
\date{\today}
	
\maketitle

\pagenumbering{roman}

\section{Revision History}

\begin{tabularx}{\textwidth}{p{3cm}p{2cm}X}
\toprule {\bf Date} & {\bf Version} & {\bf Notes}\\
\midrule
Dec 4, 2018 & 1.0 & First draft by Malavika\\
Dec 23, 2018 & 2.0 & Final draft by Malavika\\
\bottomrule
\end{tabularx}

~\newpage

\tableofcontents

%\listoftables

%\wss{Do not include if not relevant}

%\listoffigures

%\wss{Do not include if not relevant}

\newpage

\section{Symbols, Abbreviations and Acronyms}

\renewcommand{\arraystretch}{1.2}
\begin{tabular}{l l} 
	\toprule		
	\textbf{symbol} & \textbf{description}\\
	\midrule 
	T & Test\\
	\bottomrule
\end{tabular}\\
\\



Also, see the table of symbols in CA at: 
\url{https://github.com/Malavika-Srinivasan/CAS741/tree/master/docs/SRS/CA.pdf}
\\


\newpage

\pagenumbering{arabic}


This document explains the unit verification and validation plan, which aims at 
verifying the modules of \famname{} to improve the qualities such as 
reliability and correctness. It is organized into different 
sections giving a detailed description of the \famname{} in terms of its 
goals, objectives, essential qualities and test cases to verify each module 
listed in the MG. 



\section{General Information}

This section explains the summary of what is being tested in this document, the 
objectives of this document and references for this document. The qualities 
which are essential for \famname{} are correctness, reliability, 
maintainability, testability and portability.


\subsection{Purpose}

This document will summarize the plan for verification and validation of
of the modules of \famname{} in compliance with their functions and semantics 
mentioned in the MIS document found at 
\url{https://github.com/Malavika-Srinivasan/CAS741/tree/master/docs/Design/MIS/MIS.pdf}

The goal statement from the CA document is presented below.\\
\noindent ``Given the set of data points, the choice of software from 
\famname{} and the variabilities of the software, the \famname{} should:

\begin{enumerate}
	
	\item Compute the parameters of the curve which is the best possible fit 
	through the set of data points''.
\end{enumerate}


\subsection{Scope}

Some of the modules such as sequence services and plot module are not 
implemented by \famname{}, but are a part of it. The services offered by these 
modules, are implemented by python and testing them are out of scope for this 
project.

The output module of \famname{} does not compute anything. It essentially gives 
the output from the modules to the user program by using access methods from 
interpolation or regression module. It uses plot module for plotting the 
results. Hence, the testing of output module is covered by the test cases of 
interpolation and regression module.

The Input module, Interpolation module and the Regression module are tested.

\wss{What modules are you testing?}\ms{Added}

\section{Plan}

This section represents the planning phase for unit verification and validation 
plan in terms of people, tasks and tools involved in this process.
	
\subsection{Verification and Validation Team}

The verification and validation team for \famname{} includes the following 
people.
\begin{itemize}
	\item Malavika Srinivasan 
	\item Dr.\ Spencer Smith \wss{missing a space}\ms{space added}
	\item Hanane Zlitni
\end{itemize}

\subsection{Automated Testing and Verification Tools}

In this project, we will be using pytest to verify the modules of \famname{}. 
It is a testing framework available in python. The results of the test 
including the code coverage metrics will be presented in the test report. We 
will be using cov package which gives the code coverage metrics.

\subsection{Non-Testing Based Verification}

Code walkthrough and inspection was planned but due to lack of time, they are 
not conducted at this point of time.

 \wss{why is this not applicable?  You don't have to do inspections, code
  walkthroughs etc, but you should say what it isn't applicable.  If the reason
  is a lack of time, that can be the reason you give.}\ms{Explanation added}

\section{Unit Test Description}

The test cases presented here will verify the access programs of the modules 
listed in MIS. Usually, each access method will have a test case associated 
with it. If not, the reason for why the method does not have an associated test 
case is explained. The MIS can be found at 
\url{https://github.com/Malavika-Srinivasan/CAS741/tree/master/docs/Design/
	MIS/MIS.pdf}

\subsection{Tests for Functional Requirements}

In this section, we present the test cases which are related to the functional 
requirements of \famname{}. The functional requirements can be found at 
\url{https://github.com/Malavika-Srinivasan/CAS741/tree/master/docs/SRS/CA.pdf}
 
\wss{Most of the verification will be through automated unit testing.  If
  appropriate specific modules can be verified by a non-testing based
  technique.  That can also be documented in this section.}
\ms{I have parallel testing, which makes sense as a system test. But not for 
unit test.}

\subsubsection {Input Module}

In this section, we will present the test cases for each access method of the 
input module. Since all the access methods are tested, the entire module will 
be tested. The test cases for this module are selected based on the constraints 
for the input data. \\
The typical constraints on the input data for \famname{} is associated with the 
data type and length of the input data. Some of the inputs such as the degree 
of the polynomial have mathematical restrictions of being a natural number. The 
test cases listed below are designed to ensure that the input data to 
\famname{} does not violate these constraints.  


\begin{enumerate}

\item{\textbf{T1: Test case for verify input - One data point}}\\
Please refer to test case T1 in section \ref{InputTesting} in System 
verification and validation plan.
				
Test Case Derivation: We need at least two points to draw a straight line which 
is the lowest order polynomial. Only one data point does not need a curve to be 
represented. Hence, \famname{} should throw an exception.
					
\item{\textbf{T2: Test case for verify input - Length mismatch}}

Type: Automatic
					
Initial State: NA 
					
Input: t=[1,2,3] , y=[2,4] (length mismatch)
					
Output: Exception message (``t and y array must have same length.'')

Test Case Derivation: NA

How test will be performed: Pytest


\item{\textbf{T3: Test case for verify input - Type error}}\\
Please refer T2 in section \ref{InputTesting} in System verification and 
validation plan.
					

Test Case Derivation: It is necessary that the input arrays contains only 
numbers.



\item{\textbf{T4: Test case for verify degree - value error}}

Type: Automatic

Initial State: NA 

Input: deg = 1.5 

Output: Exception message (``Degree cannot be a real number'')

Test Case Derivation: NA

How test will be performed: Pytest

%%%%%%%%%%%%%%%%%%%%%%%%%%%%%%%%%%%%%%%%%%%%%%%%%%%%%%%%%%%%%%%%%5

    
\subsubsection{Interpolation module}

In this section, we will represent the test cases for each access program 
present in the interpolation module. This module has access programs to find 
the coefficients of the interpolated curve through a set of points $(t_i,y_i)$ 
for $i = 0\ \text{to}\ n$ and find the value of the interpolating polynomial at 
a given `$t$' value.

\item{\textbf{T5: Test case for interpMonomial}}

Please refer to test cases T4 and T5 in section \ref{InterpolationTesting} in 
System verification 
and 
validation plan.


\item{\textbf{T6: Test case for interpLagrange}}

Please refer to test cases T6 and T7 in section \ref{InterpolationTesting} in 
System verification 
and 
validation plan.


\item{\textbf{T7: Test case for interpNewton}}

Please refer to test cases T8 and T9 in section \ref{InterpolationTesting} in 
System verification 
and 
validation plan.

\item{\textbf{T8: Test case for interpHermiteCubic}}

Please refer to test cases T10 and T11 in section \ref{InterpolationTesting} in 
System verification and validation plan.


\item{\textbf{T9: Test case for interpBSpline}}

Please refer to test case T12 in section \ref{InterpolationTesting} in System 
verification and validation plan.


\item{\textbf{T10: Test case for evalMonomial}}

Please refer to test case T13 and T14 in section \ref{InterpolationTesting} in 
System verification and validation plan.



\item{\textbf{T11: Test case for evalLagrange}}


Please refer to test case T15 and T16 in section \ref{InterpolationTesting} in 
System 
verification and validation plan.


\item{\textbf{T12: Test case for evalNewton}}

Please refer to test case T17 and T18 in section \ref{InterpolationTesting} in 
System 
verification and validation plan.


\item{\textbf{T13: Test case for evalHermiteCubic}}

Please refer to test case T19 and T20 in section \ref{InterpolationTesting} in 
System 
verification and validation plan.


\item{\textbf{T14: Test case for evalBSpline}}

Please refer to test case T21 in section \ref{InterpolationTesting} in System 
verification and validation plan.




%%%%%%%%%%%%%%%%%%%%%%%%%%%%%%%%%%%%%%%%%%%%%%%%%%%%%%%%%%%%%%%%%%%%%%%%%%%%%%

\subsubsection{Regression Module}

In this section, we will represent the test cases for each access program 
present in the regression module. This module has access programs to find 
the parameters of the best fit curve through a set of points $(t_i,y_i)$ 
for $i = 0\ \text{to}\ n$ and find the value of the polynomial at a given `$t$' 
value.



\item{\textbf{T15: Test case for regNormalEq}}

Please refer to test cases T22, T23 and T24 in section \ref{ResgressionTesting} 
in 
System verification 
and 
validation plan.


\item{\textbf{T16: Test case for regAugSys}}

Please refer to test cases T25 and T26 in section \ref{ResgressionTesting} in 
System verification 
and 
validation plan.


\item{\textbf{T17: Test case for regOrthogonalTrans}}

Please refer to test cases T27 and T28 in section \ref{ResgressionTesting} in 
System verification 
and 
validation plan.

\item{\textbf{T18: Test case for evalReg}}

Please refer to test case T29 in section \ref{ResgressionTesting} in System 
verification and validation plan.


%#######################################################



\subsection{Tests for Nonfunctional Requirements}

Please see section \ref{NFRTesting} in System verification and validation plan.

\subsection{Traceability Between Test Cases and Modules}


The following table shows the traceability mapping for test cases and the 
modules. All the modules except the ones listed as out of scope are covered and 
every access program in the module is tested. This may be considered as a proof 
for coverage of modules.

\begin{table} [H]
	\caption{Modules Traceability Matrix}
	\label{Table:Table_Traceability}  
	\begin{tabular}{|c|p{5cm}|}
		\hline	
		\textbf{Test Number} & \textbf{Modules} \\
		\hline 
		T1&  Input      \\ \hline
		T2&  Input      \\ \hline
		T3&  Input     \\ \hline
		T4&  Input     \\ \hline
		T5&  Interpolation      \\ \hline
		T6&  Interpolation     \\ \hline
		T7&  Interpolation     \\ \hline
		T8&  Interpolation      \\ \hline
		T9&  Interpolation     \\ \hline
		T10& Interpolation      \\ \hline
		T11& Interpolation      \\ \hline
		T12& Interpolation     \\ \hline
		T13& Interpolation     \\ \hline
		T14& Interpolation      \\ \hline
		T15& Regression     \\ \hline
		T16& Regression     \\ \hline
		T17& Regression      \\ \hline
		
	\end{tabular}\\
\end{table}

\wss{Please give the names of the modules so that they can be compared to the
  information (which you will add) in the Scope section.}\ms{Added module name}





\bibliographystyle{plainnat}

\bibliography{SRS}

\newpage

\section{Appendix}

NA

\subsection{Symbolic Parameters}

NA


\end{enumerate}

\end{document}