\documentclass{article}

\usepackage{tabularx}
\usepackage{booktabs}

\title{CAS 741: Problem Statement\\ Scientific Data Processing}

\author{Malavika Srinivasan and sriniva}

\date{Sep 15, 2018}

\input{../Comments}

\begin{document}

\maketitle

\begin{table}[hp]
\caption{Revision History} \label{TblRevisionHistory}
\begin{tabularx}{\textwidth}{llX}
\toprule
\textbf{Date} & \textbf{Developer(s)} & \textbf{Change}\\
\midrule
Sep 15, 2018 & Malavika Srinivasan & Problem statement creation\\
... & ... & ...\\
\bottomrule
\end{tabularx}
\end{table}

Scientific Computation (SC) is the collection of tools, techniques, and
theories that are required to solve problems in the field of science and
engineering using computer-based mathematical models. The source data for
scientific computation problems are usually from a data acquisition system
which is used for conducting experiments in a laboratory setup. The large set
of data across the entire function (such as time and temperature) in an
experiment is usually complex to analyze and require segmenting and
curve-fitting.

The purpose of this software is to develop a general purpose tool for fitting
the experimental data across a 1D function to enable the data processing in a
more simpler fashion. In other words, the software is intended to be used as a
tool to fit the data and compute gradients. In this software, regression and
interpolation are the two techniques used for data fitting.

Interested stakeholders in this project may include researchers in industrial
and academic set-up, students, technicians and those who deal with processing
of a large set of data to obtain critical information within a dataset. The
processing encoded into the current software will be suitable for any
scientific application which will require scientific data processing like
obtaining a fit for the data, interpolating across the domain and computing
gradients. This software can be run on a variety of personal desktop and laptop
computers using Linux, Windows, or MacOS.


%Put your problem statement here.  Comments to you can be added, like this:

%\wss{comment}

%You can also leave comments for yourself, like this:

%\an{comment}

\end{document}